\documentclass[11pt]{article}
\usepackage{latexsym}
\usepackage[vlined,ruled]{algorithm2e}
\usepackage{amsmath}
\usepackage{amssymb}
\usepackage{amsthm}
\usepackage{epsfig}
\usepackage{psfrag}
\usepackage{color}
\input{rgb}
\newcommand{\handout}[5]{
  \noindent
  \begin{center}
  \framebox{
    \vbox{
      \hbox to 5.78in { {\bf CS345 : Algorithms II } \hfill #2 }
      \vspace{4mm}
      \hbox to 5.78in { {\Large \hfill #5  \hfill} }
      \vspace{2mm}
      \hbox to 5.78in { {\em #3 \hfill #4} }
    }
  }
  \end{center}
  \vspace*{4mm}
}

\newcommand{\lecture}[4]{\handout{#1}{#2}{#3}{}{#1}}


\newcommand{\TT}[1]{\textsc{#1}}
\newtheorem{theorem}{Theorem}
\newtheorem{corollary}[theorem]{Corollary}
\newtheorem{lemma}[theorem]{Lemma}
\newtheorem{observation}[theorem]{Observation}
\newtheorem{proposition}[theorem]{Proposition}
%\newtheorem{problem}[theorem]{\color{darkred}{\bf Problem}}
\newtheorem{exercise}[theorem]{\color{DarkBlue}{\bf Exercise}}
\newtheorem{problem}[theorem]{\color{darkred}{\bf Problem}}
\newtheorem{definition}[theorem]{Definition}
\newtheorem{claim}[theorem]{Claim}
\newtheorem{fact}[theorem]{Fact}
\newtheorem{assumption}[theorem]{Assumption}

\topmargin 0pt
\advance \topmargin by -\headheight
\advance \topmargin by -\headsep
\textheight 8.9in
\oddsidemargin 0pt
\evensidemargin \oddsidemargin
\marginparwidth 0.5in
\textwidth 6.5in

\parindent 0in
\parskip 1.5ex
\title{Assignment 5}
\author{Satvik Chauhan (Y9521),Pankaj More (Y9402)}
\begin{document}
\maketitle
\section*{Question 1}
\begin{claim}
Let $d_i(v)$ denotes the distance from s to v at the end of ith iteration, then 
$d_j(v) \geq d_i(v) $ for all $j > i$ and v $\in$ V 
\end{claim}
\begin{proof}
	Let the above assertion does not hold for ith iteration for some $i > 0$. So let $d_i(v) < d_{i-1}(v)$ for some
vertex v. Furthermore, let v be such a vertex with least $d_i(v)$, that is, for each x = v with $d_i(x) < d_i(v)$,
it must be that $d_i(x) \geq d_{i-1}(x)$. 
	Since d(v) changes after ith iteration , there must be at least one edge inserted during the ith iteration that now belongs to the shortest path from s to v.
	Let that edge be (y,x). Before ith iteration , the shortest path to v was from s---x and x---v .
	After ith iteration , it has become s----y , y-x , x---v. 
\[d_{i-1}(v) = d_{i-1}(x) + \delta(x,v) \]
\[d_{i}(v) = d_{i}(x) + \delta(x,v) \]

Since ,
\[d_i(v) < d_{i-1}(v) \]
we get,
\[d_i(x) < d_{i-1}(x) \] 
(contradiction)

\end{proof}
\pagebreak
\begin{claim}
Let (u, v) be an edge which disappears after some ith iteration, and then reappears after some jth iteration. 
$d_j(u) > d_{i-1}(u)$ . 
\end{claim}
\begin{proof}
Since (u,v) reappears after jth iteration , it must be the case that (v,u) belonged to the shortest s-t path after (j-1)th iteration. Similarly , since (u,v) disappeared after ith iteration , it must belong to the shortest s-t path after (i-1)th iteration.
\[d_{j-1}(v) = d_{j-1}(u) - 1 \]
\[d_{j-1}(v) < d_{j-1}(u) \leq d_j(u) \]
\[d_{i-1}(u) = d_{i-1}(v) - 1 \]
\[d_{i-1}(u) < d_{i-1}(v) \leq d_i(v) \]

\[d_j(u) > d_{i-1}(u) \]

 
\end{proof}

\begin{claim}
The modified FF algorithm has running time $O(m^2n)$.
\end{claim}
\begin{proof}
	Since the shortest s-t path can be computed using BFS in O(m+n) time and each vertex has at least one incident edge , the time complexity of finding shortest path is O(m) . The rest of the iteration is O(m). So total time complexity of each iteration is O(m).
	Consider an edge (u,v). After it disappears and reappears once , $d(u)$ increases by at least 1. d(u) can vary from 0 to n.    So an edge can disappear only O(n) times.Since there are m edges , total number of disappearance is O(mn).In each iteration of the algorithm , at least 1 edge(the bottleneck edge) disappears.Hence , total number of iterations cannot exceed O(mn). 
	Therefore, total running time of the algorithm = $O(m^2n)$
\end{proof}
\pagebreak
\section*{Question 2}
\begin{claim}
All the edges removed by the attacker will belong to some min-cut.
\label{mincut}
\end{claim}
\begin{proof}
It is given that each node lies on atleast one path from $s-t$.
From the max-flow , min-cut theorem , the size of min-cut is k. Consider a cut as follows :

The cut divides the graph into two partitions $s \in A$ and $t \in A'$ , such that for any edge $(u,v)$ removed $u \in A$ and $v \in A'$. Also $\forall v' \in A'$ , $t$ is reachable from $v'$ and $\forall u' \in A $ , $u'$ is reachable from $s$ , as no such edge $(x,y)$ is removed whose both end points lie either in $A$ or in $A'$.

If the size of this cut is greater than k , then $\exists (x,y) \in E$ not removed by the attacker such that $x \in A$ and $y \in A'$.
since $x$ is reachable from $s$ and $t$ is reachable from $y$ , and edge $(x,y)$ is not removed , so $t$ is reachable from $s$ , which leads to contradiction. 

Hence the size of this cut is k. Thus it is a min-cut.
\end{proof}

\begin{claim}
Any edge $(u,v)$ is removed by the attacker if and only if $ping(u)$ is true and $ping(v)$ is false. 
\label{ping}
\end{claim}
\begin{proof}
if part :

$(u,v)$ is removed by the attacker so $u \in A$ and $v \in A'$.

$ping(u)$ is true follows trivially from the fact that $u \in A$.

If all the edges of the above min-cut are removed then there is no path from any vertex in $A$ to any vertex in $A'$. 
As $s \in A$ and $v \in A'$, so $ping(v)$ is false.

only if part :

Suppose $(u,v)$ be such edge with $ping(u)$ = true and $ping(v)$ = false.

$v$ can not belong to $A$ as all vertices in $A$ are reachable from $s$. Similarly $u$ can not belong to $A'$ , if $u$ belonged to $A'$ then there is a path from $u$ to $t$ from the property of $A'$ and there is a path from $s$ to $u$ as $ping(u)$ is true , so there is a path from $s$ to $t$ which is contradiction. 

Thus $u \in A$ and $v \in A'$ , since this edge forms a part of the min-cut edges so it would have been removed by the attacker.

\end{proof}

\begin{algorithm}[h]
$G'$ $\leftarrow$ $G$ \;
\For {$i=1$ to k}{
	Let $\lbrace x_1=s,x_2,\ldots x_m=t \rbrace$ be any $s-t$ path in $G'$. \;
	$L = 1$ \;
	$R = m$ \;
	$found = False$ \;
	\While{$L \leq R$ and $found = False$}
	{
	$mid \leftarrow (L+R)/2$ \;
	\eIf{$ping(x_{mid})$ == $True$ and $ping(x_{mid+1})$ == $False$}
	{
	$found=True$ \;
	remove edge $(x_{mid},x_{mid+1})$ from $G'$ \;
	}
	{
	\eIf{$ping(x_{mid})==True$}
	{
	$L \leftarrow mid+1$ \;
	}
	{
	$R \leftarrow mid-1$ \;
	}
	}
	}
}
do a dfs on $G'$ from $s$ to find all vertices reachable from $s$ which will form the set $A$. \;
$V-A = A'$ which is the set of vertices not reachable from $s$. \;
\caption{{\em Find\_Reachability($G(V,E)$)}:~
Algorithm to report vertices not reachable from $s$.
}
\label{Algorithm}
\end{algorithm}
\subsection*{Order of ping requests}
Since length of any $s-t$ path can not be greater than $n$ and we are issuing requests in binary search fashion. 
So number of ping requests in the inner while loop is $O(log n)$.
Since for loop is repeated k times, so total number of ping requests are $O(k logn)$.

\subsection*{Correctness of Algorithm}
\begin{claim}
Every $s-t$ path we are guaranteed to find an edge $(u,v)$ which is removed by the attacker.
\end{claim}
\begin{proof}
Let there be no such edge , then in the new Graph after removing the attacked edges , all the edges of this path exist. So there is a $s-t$ path which leads to contradiction. 
\end{proof}

So in every iteration , the path we find , we are guaranteed to find an edge $(u,v)$ which is present in this path and is removed by the attacker. This edge is found by the binary search algorithm using the claim $\ref{ping}$. So every iteration we are removing one edge which is actually attacked by the attacker. 

We are also guaranteed to find a $s-t$ path in every iteration as $k$ is the minimum number of edges to be removed to separate $s$ and $t$ and we are running for loop only $k$ number of times. So after $k$ iterations we find $k$ edges which are actually attacked by the attacker. So the graph $G'$ consist of only edges actually present in the network after the attack. Now simple dfs from $s$ finds the vertices actually reachable from $s$ which forms the set $A$ . $V-A=A'$ forms the set of vertices not reachable from $s$.

\end{document}
