\documentclass[11pt]{article}
\usepackage{latexsym}
\usepackage[lined,boxed]{algorithm2e}
\usepackage{amsmath}
\usepackage{amssymb}
\usepackage{amsthm}
\usepackage{epsfig}
\usepackage{psfrag}
\usepackage{color}
\input{rgb}
\newcommand{\handout}[5]{
  \noindent
  \begin{center}
  \framebox{
    \vbox{
      \hbox to 5.78in { {\bf CS345 : Algorithms II } \hfill #2 }
      \vspace{4mm}
      \hbox to 5.78in { {\Large \hfill #5  \hfill} }
      \vspace{2mm}
      \hbox to 5.78in { {\em #3 \hfill #4} }
    }
  }
  \end{center}
  \vspace*{4mm}
}

\newcommand{\lecture}[4]{\handout{#1}{#2}{#3}{}{#1}}


\newcommand{\TT}[1]{\textsc{#1}}
\newtheorem{theorem}{Theorem}
\newtheorem{corollary}[theorem]{Corollary}
\newtheorem{lemma}[theorem]{Lemma}
\newtheorem{observation}[theorem]{Observation}
\newtheorem{proposition}[theorem]{Proposition}
%\newtheorem{problem}[theorem]{\color{darkred}{\bf Problem}}
\newtheorem{exercise}[theorem]{\color{DarkBlue}{\bf Exercise}}
\newtheorem{problem}[theorem]{\color{darkred}{\bf Problem}}
\newtheorem{definition}[theorem]{Definition}
\newtheorem{claim}[theorem]{Claim}
\newtheorem{fact}[theorem]{Fact}
\newtheorem{assumption}[theorem]{Assumption}

\topmargin 0pt
\advance \topmargin by -\headheight
\advance \topmargin by -\headsep
\textheight 8.9in
\oddsidemargin 0pt
\evensidemargin \oddsidemargin
\marginparwidth 0.5in
\textwidth 6.5in

\parindent 0in
\parskip 1.5ex

\begin{document}
\section*{Question 1}

\subsection*{Design of data structure:}
\begin{itemize}
\item Given the set P of points in 2-D space , let us assume that all the points in P have pairwise distinct coordinates. \\
\item If P is empty , then the binary search tree is an empty leaf . Otherwise,
\item let $p_{max}$ = point with the largest y-coordinate in P
\item $x_{mid}$ = median of x coordinates of points in $P \setminus \{p_{max}\}$
\item $P_{l}$ = \{$p \in P \setminus \{p_{max}\} : x \leq x_{mid}$\}
\item $P_{r}$ = \{$p \in P \setminus \{p_{max}\} : x > x_{mid}$\}
\item Point $p_{max}$ is stored at the root
\item The left subtree is a binary search tree of $P_{l}$
\item The right subtree is a binary search tree of $P_{r}$
\end{itemize}

In other words , our new binary search tree acts like a max-heap with respect to y-coordinates and a standard binary search tree with respect to x-coordinates.

\begin{algorithm}
\SetKwFunction{ConstructTree}{ConstructTree}
\SetKwData{Left}{v.left}
\SetKwData{Right}{v.right}
\SetKw{Return}{return}
\ConstructTree{P} { \\
\Indp
\If{$P = \phi$} {\Return nil \;}

v.data $\leftarrow$ point with largest y-coordinate in P ; ~~~~$\backslash \backslash$ {\tt{$O(n)$}} \\
v.mid $\leftarrow$ median of x-coordinates in P ; ~~~~~~~~~~~~~~$\backslash \backslash$ {\tt{$O(n)$}} \\
$P_{l} \leftarrow $ \{$p \in P \setminus \{p_{max}\} : x \leq v_{mid}$\} ; ~~~~~~~~~~~~~~~~~~~~~$\backslash \backslash$ {\tt{$O(n)$}} \\
$P_{r} \leftarrow $ \{$p \in P \setminus \{p_{max}\} : x > v_{mid}$\} ; ~~~~~~~~~~~~~~~~~~~~~$\backslash \backslash$ {\tt{$O(n)$}} \\
\Left $\leftarrow$ \ConstructTree{$P_{l}$} ; ~~~~~~~~~~~~~~~~~~~~~~~~~~~~$\backslash \backslash$ {\tt{$T(n/2)$}} \\
\Right $\leftarrow$ \ConstructTree{$P_{r}$} ; ~~~~~~~~~~~~~~~~~~~~~~~~~~$\backslash \backslash$ {\tt{$T(n/2)$}} \\
return v; 
\Indm
}
\caption{\textsc{ConstructTree}(P)}
\end{algorithm}


\subsection*{Space Complexity:}  
\begin{itemize}
\item Size of each node is constant since it has at most 4 fields i.e data,mid,left and right
\item Every node in the tree maps to exactly one point in the set P
\item Total number of points in set P is n
\item Hence , total space occupied by all the nodes is O(n)
\end{itemize}

\subsection*{Preprocessing Time:}

Preprocessing time = Time taken to execute the ConstructTree function
\[T(n) \leq 2T(\frac{n}{2}) + cn \]
\[T(0) \leq c\]

\begin{align*}
T(n) & \leq 2T(n/2) + cn \\
&\leq 2(2T(n/4) + cn/2) + cn \\
&\leq 2^{logn} * T(0) + cn * logn \\
&\leq cn + cnlogn   \\
&\leq O(nlogn) \\
\end{align*}


Hence, preprocessing time is O(nlogn)

\subsection*{Query Time:}

\begin{algorithm}
\SetKwFunction{RH}{reportHeap}
\tcc{Let $v_{split}$ is the least common ancestor of $x_1$ and $x_2$}
Start from the root node v and find the first split-node $v_{split}$ such that $v_{split}.mid \in [x_1,x_2]$
\ForEach{node u on the path from node $v_{split}$ to node containing median $x_1$ and to node containing median $x_2$}{
    \If{u.mid $\in [x_1,x_2] \times [y_0,\infty]$}{
	report it in S \;    
    }
} 
\ForEach{node u on the path upto node containing median $x_1$ in the left subtree of $v_{split}$}{
    \If{the path goes left at u}{
    \RH{u.right} \;
    }
}
\ForEach{node u on the path of node containing median $x_2$ in the right subtree of $v_{split}$}{
    \If{the path goes right at u}{
    \RH{u.left} \;
    }
}
\RH{v} { \\
\tcc{Reports all the points in tree rooted at v (as heap) whose y coordinates are greater than or equal to $y_0$} 
\Indp
\eIf{v.data.y $< y_0$} {return \;}
{report it in S \;
\RH{v.left} \;
\RH{v.right} \;
\Indm
}
}
\caption{\textsc{3-SidedRangeQuery}(P,$x_1$,$x_2$,$y_0$)}
\end{algorithm}

\begin{itemize}
\item Time required for each query is the sum of time required to visit $|S|$ points in set S and the time spent in visiting (and discarding) the points which do not belong to S. 
\item The set of points visited by the algorithm is the union of 4 non-disjoint subsets : -
\begin{itemize}
\item S = $\{ (x,y) \in P : x_1 \leq x \leq x_2 and y \geq y_0 \}$ 
\item $X_L$ = $\{ (x,y) \in P : x \leq x_1 \}$
\item $X_G$ = $\{ (x,y) \in P : x \geq x_2 \}$
\item $Y_L$ = $\{ (x,y) \in P : y \leq y_0 \}$
\end{itemize}
\item Points from set S are reported whereas points from other 3 sets are discarded
\item S requires O($|S|$) time
\item $X_L$ and $X_R$ requires O(logn) time since there are O(logn) nodes in the path from $x_1$ to $x_2$
\item For $Y_L$ ,
\begin{itemize}
\item If the root of a subtree has $y < y_0$ , we can completely discard that subtree.
\item The parent node of this subtree must have $y \geq y_0$ and it must belong to $S \cup X_L \cup X_G$
\item The candidate subtrees for $Y_L$ can be discarded from its corresponding parent nodes in constant time.
\item Number of such parent nodes can be O($|S| + logn$)
\item Hence, $Y_L$ requires O($|S| + logn$) time 
\end{itemize}
\item Hence , total query time is O(logn + $|S|$) 
\end{itemize}
  



\pagebreak

\section*{Question 2}
We will define our own coordinate system as follows . 
\begin{itemize}
\item Choose any arbitrary point on the circle which is not the endpoint of any chord .
\item This is called zero . Now travel clockwise from this point .
\item The degrees we travel from this point to the concerned point is the value of that point . 
\item So minimum value can be 0 and maximum can be 360. 
\end{itemize}
The conversion from the given system to ours can be done in $O(1)$ time for each point . Hence total $O(n)$ time .\\
Now algorithm to give the no. of intersections is : \\
\begin{itemize}
\item We assume that point are given in some coordinate system with $(x_i,y_i)$ as the two endpoints of the $i^{th}$ chord.
\item We convert each point into the coordinate system described above and for a chord the point which has lower value (closer to the zero point) is marked as the start point of the chord and the other as the corresponding end point.
This takes O(n) time.
\item Now we sort all the points (including all the start and end points).
\item These nodes (points) contain information about which chord they belong to and also a pointer to the other end of the chord. 
\item For each point taken from the sorted list we do the following :
	\begin{itemize}
	\item If it is a start point we add it to the BST where each node is augmented with the size of the subtree rooted at that node (We have to increase the size field of all the nodes along the path where we insert the given node). This takes O(log n) time because height of the BST in the worst case containing all the start nodes can not be greater than log n because there are only n chords and insertion in such a BST can not take more than O(log n).
	\item If it is an end point then we count the number of points greater than the corresponding start point (which can be found as we are keeping the pointer for it from the end point). We travel the BST from the root till we find the start point (maximum O(log n) steps) . While travelling along the path if we move left then we add (1 + size of right subtree which is stored in the augmented BST and can be found in O(1) time) and if we move right then we dont add anything. Thus the number of points greater than the given point can be found in O(log n) time. 
    \item This number correspond to the number of chords intersecting the given chord.
    \item Delete the start point from the BST. This takes O(log n) time (We have to decrease the size field accordingly for the nodes on the path ). This is same as BST (augmented with size field) discussed in the class.
	\end{itemize}
	We sum all the points of intersections . Since we remove the start point from the the BST after counting its intersections we are making sure that we count each intersection only once and not twice. 
	Since we are doing this for n chords thus total O(n(log n)) time. 
\item Total time = $O(n)$(for coordinate conversion) + $O(nlog n)$(for sorting) + $O(nlog n)$(for the final counting) = O(n(log n)).
\end{itemize}
\begin{algorithm}
\For{$i=1$ to $n$}{
        Convert the point into above coordinate system and store into $(s_{i},e_{i})$ such that $s_{i} < e_{i}$.
        Each point contains pointer to its corresponding other point of the chord \;
    } $\backslash \backslash$ {\tt{$O(n)$ for the loop}} \\
$count \leftarrow  0$ \;
$B \leftarrow$ Empty BST where each node augmented with the size field \;
$Points \leftarrow sort (<s_1,s_2,\ldots,s_{n},e_1,e_2,\ldots,e_{n}>)$;  ~~~~~~~~~~~~~~~~~~~~~~~~~~~~~~$\backslash \backslash$ {\tt{$O(nlog n)$}} \\
\ForEach{point p in Points}{
    \eIf{p is a start point}{
    	insert p into the augmented BST B;~~~~~~~~~~~~~~~~~~~~~~~~~~~~~~~~~~~~~~$\backslash \backslash$ {\tt{$O(log n)$}}}
    {  
    	$count \leftarrow count$ + points greater than start point in the BST ;~~~~~$\backslash \backslash$ {\tt{$O(log n)$}} \\
    	delete corresponding startpoint from B ;~~~~~~~~~~~~~~~~~~~~~~~~~~~~~~~~~~$\backslash \backslash$ {\tt{$O(log n)$}} \\  
    	}
    }$\backslash \backslash$ {\tt{$O(nlog n)$ for the loop}}   \\                                                                   
    return $count $
\caption{\textsc{findintersection(\(<(x\sb1,y\sb1),(x\sb2,y\sb2),\ldots ,(x\sb{n},y\sb{n})>\)) }}
\label{chord_intersection}
\end{algorithm}
\pagebreak
	
\section*{Question 3}
\begin{lemma} If an element u is k-majority in a multiset. Then removing k distinct element , u is still k-majority in the remaining multiset. 
\end{lemma}
\begin{proof}
Case 1 : u is not in the k elements removed .  Then $Count(u)>\frac{n}{k}>\frac{n-k}{k}$ . Hence it holds. \\
Case 2 : u is in the k elements removed . Then 
\[Count(u)+1>\frac{n}{k}\]
\[Count(u)>\frac{n-k}{k}\]
Hence it is still k majority in remaining n-k elements.
\end{proof}

\begin{algorithm}
$i\leftarrow 0$\;
$d\leftarrow 0$;    ~~~~~$\backslash \backslash${\tt{Number of distinct values in t}} \\
$t\leftarrow \lbrace \rbrace$\;
\For{$i=1$ to $n$}{
\eIf{$A[i] \notin t$}{
	$t \leftarrow t\cup \lbrace (b[i],1) \rbrace$; ~~~~~~~~~~~$\backslash \backslash${\tt{Initial count for element is assigned 1}}
	$d \leftarrow d+1$ \;
	\If{$d=k$}{
		decrease count of all elements in t by 1 and update d \;
	}}
	{
     increase count of $A[i]$ in t by 1\;
      }
}
Count the occurrence of the elements in t in $A$ and if any of it occurs more than $n/k$ times, then return that element else return NIL.
\caption{\textsc{finding\_K-majority\_element}($A[1..n]$)}
\label{majority-element-algo}

\end{algorithm}
\subsection*{ Implementation Details and Time complexity :} 
t in the above algorithm is imlemented as AVL tree. The number of elements in t never exceed k so it takes O(k) size. \\ \\
\textbf{Operations Performed}
\begin{itemize}
\item $t \leftarrow \lbrace \rbrace$ . Performed once . Complexity is $O(1)$.
\item Search A[i] in t. Performed n times . Complexity is $n*O(log k)$ = $O(nlog k)$
\item Insert an element or increase the count of an element by 1. Both the operations together is performed n times because one belongs to if and other belongs to else , so in each iteration  only one of them is executed. Both takes O(log k) time. So for n iterations Complexity is $O(nlog k)$.
\item Decrease count of all k elements by 1 and update d. The thing to note is this operation can be performed atmost $\frac{n}{k}$ times . It requires traversal of the whole tree and deleting a node if its count becomes 0 . So for each element it takes $O(log k)$ time . So $O(k log k)$ time for k elements . So complexity is $O(\frac{n}{k}klog k)$ = $O(nlog k)$. 
\item Checking for an element whether it is k-majority or not. It is performed for last k-1 elements . We have to do it  $O(n log k)$ for overall algorithm to be $O(n log k)$. It is done as follows:
\begin{itemize}
\item Construct a BST (height balanced) for these k-1 elements. This takes $O(klogk)$ . Since k is less than n . So it is less than $O(nlog k)$.
\item For each element of the original array , search it in the BST . If it appears then increase the count of that element by 1. This takes $O(logk)$ for a single element. So total $O(n log k)$ time. 
\item Now find the element with maximum count in the BST . This requires scanning the whole BST which is $O(k)$ . If the count is greater than $\frac{n}{k}$ then returnt the element . 
\end{itemize} 
\end{itemize}
So the net complexity of the algorithm is $O(nlog k)$.

\subsection*{Loop invariant}
If u is the k-majority in $A$ then it is also k majority in $A'=B \cup \lbrace A[i+1],A[i+2],\ldots A[n-1] \rbrace$ ($\cup$ is concatenation operation or union without removing duplicates) where \[B = \lbrace \underbrace{e,e \ldots ,e}_{c \; times} | \forall (e,c) \in t \rbrace \] after $i^{th}$ iteration.
\begin{proof}
By induction \\
\textbf{Base Case :}  $A'=A$ . Hence it holds trivially. \\
\textbf{Induction Hypothesis :} u is k-majority in $A'$ after $i^{th}$ interation. \\
\textbf{Induction Step :} At $i+1^{th}$ iteration : \\
\begin{itemize}
\item if $A[i+1] \notin t$ then $ t \leftarrow t \cup \lbrace (A[i+1],1) \rbrace$ \\
\[A'=B\cup \lbrace A[i+1],A[i+2],\ldots A[n-1] \rbrace\]
\[A'_{new} = (B\cup \lbrace A[i+1]\rbrace)\cup \lbrace A[i+2],A[i+3],\ldots,A[n-1]\rbrace=A'\]
Hence u is still k-majority in $A'_{new}$.
\item if $d=k$ after previous step then count of all k elements is decreased by 1. So k distinct elements are removed from $A'_{new}$. By lemma 1 u is still k-majority in the remaining multiset.  
\item if $A[i+1] \in t$ then count of $A[i+1]$ is increased by 1 . \\
\[A'=B\cup \lbrace A[i+1],A[i+2],\ldots A[n-1] \rbrace\]
\[A'_{new} = (B\cup \lbrace A[i+1]\rbrace)\cup \lbrace A[i+2],A[i+3],\ldots,A[n-1]\rbrace=A'\]
Hence u is still k-majority element in $A'_{new}$.

\end{itemize}
\end{proof}

\subsection*{Correctness of Algorithm}
When algorithm terminates then 
\[A'=B \cup \lbrace \rbrace\]
So by loop invariant k-majority element of A is also k majority of B. Since we are checking for all the elements of B in A for k-majority . Hence if k-majority exist then the checking step returns it otherwise it returns nil. 
\pagebreak



\section*{Question 4}
Let : 
\[r = x * z \] is the convolution of 2 vectors x and z and \\
\[s = y * z \] is the convolution of 2 vectors y and z such that
\[x = (x_{0},x_{1},...,x_{n-1})\]
\[y = (y_{0},y_{1},...,y_{n-1})\]
\[z = (z_{0},z_{1},...,z_{n-1})\]
\[x_{i} = q_{i+1}\]
\[y_{i} = q_{n-i}\]
\[z_{j} = 1/(j+1)^2\]

\subsection*{Claim:} \[F_j = Cq_j[r_{j-2} - s_{n-j-1}]\]

\subsection*{Proof:}
\begin{align*}
r_{j-2} & = \sum_{l+m=j-2} x_lz_m \\
&= \sum_{l+m=j-2} \frac{q_{l+1}}{(m+1)^2} \\
&= \frac{q_1}{(j-1)^2} + \frac{q_2}{(j-2)^2} + ... + \frac{q_{j-1}}{1^2} \\
&= \sum_{i<j} \frac{q_i}{(j-i)^2} \\
s_{n-j-1} & = \sum_{l+m=n-j-1} y_lz_m \\
&= \sum_{l+m=n-j-1} \frac{q_{n-l}}{(m+1)^2} \\
&= \frac{q_n}{(n-j)^2} + \frac{q_{n-1}}{(n-j-1)^2} + ... + \frac{q_{j+1}}{1^2} \\
&= \sum_{i>j} \frac{q_i}{(j-i)^2} \\ 
\end{align*}

Hence , 
\[F_j = Cq_j[\sum_{i<j} \frac{q_{i}}{(j-i)^2} - \sum_{i>j} \frac{q_{i}}{(j-i)^2}  ]\] 
\\
Note: We assume $r_j = 0$ for $j < 0$ and same in the case of $s_j$.

\subsection*{Runtime analysis:}
Using the result of the upper bound of running time of the algorithm to compute the convolution of 2 vectors of degree n in O(nlogn) , we can compute the vectors r and s in O(nlogn) time.
For any particular value of j in $F_j$ , we only have to compute one subtraction and two multiplication which is constant time. Hence computing $F_j$ for all values of j is O(n) time. \\
Therefore , total running time of the above algorithm is O(nlogn).


\end{document}
