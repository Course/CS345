\documentclass{article}
\usepackage{graphicx}
\usepackage{wrapfig}
\usepackage{multicol}
\usepackage{epsfig}
\usepackage{amsmath}
\textwidth 6in
\addtolength{\oddsidemargin}{-0.5in}
\textheight 9in
\addtolength{\topmargin}{-0.5in}
\setlength{\parindent}{0pt}
\setlength{\parskip}{0.5cm}
\setlength{\columnsep}{5cm}
\topskip 0.0in
%\pagestyle{empty}
\thispagestyle{empty}
\title{Assignment 1}
\author{Satvik Chauhan (Y9521),Pankaj More (Y9402)}
%\date{August 04, 2004}
\date{}


\begin{document}
\maketitle
\section{Question 4}
Let : 
\[r = x * z \] is the convolution of 2 vectors x and z and \\
\[s = y * z \] is the convolution of 2 vectors y and z such that
\[x = (x_{0},x_{1},...,x_{n-1})\]
\[y = (y_{0},y_{1},...,y_{n-1})\]
\[z = (z_{0},z_{1},...,z_{n-1})\]
\[x_{i} = q_{i+1}\]
\[y_{i} = q_{n-i}\]
\[z_{j} = 1/(j+1)^2\]

\subsection*{Claim:} \[F_j = Cq_j[r_{j-2} - s_{n-j-1}]\]

\subsection*{Proof:}
\begin{align*}
r_{j-2} & = \sum_{l+m=j-2} x_lz_m \\
&= \sum_{l+m=j-2} \frac{q_{l+1}}{(m+1)^2} \\
&= \frac{q_1}{(j-1)^2} + \frac{q_2}{(j-2)^2} + ... + \frac{q_{j-1}}{1^2} \\
&= \sum_{i<j} \frac{q_i}{(j-i)^2} \\
s_{n-j-1} & = \sum_{l+m=n-j-1} y_lz_m \\
&= \sum_{l+m=n-j-1} \frac{q_{n-l}}{(m+1)^2} \\
&= \frac{q_n}{(n-j)^2} + \frac{q_{n-1}}{(n-j-i)^2} + ... + \frac{q_{j+1}}{1^2} \\
&= \sum_{i>j} \frac{q_i}{(j-i)^2} \\ 
\end{align*}

Hence , 
\[F_j = Cq_j[\sum_{i<j} \frac{q_{i}}{(j-i)^2} - \sum_{i>j} \frac{q_{i}}{(j-i)^2}  ]\] 
\\
Note: We assume $r_j = 0$ for $j < 0$ and same in the case of $s_j$.

\subsection*{Runtime analysis:}
Using the result of the upper bound of running time of the algorithm to compute the convolution of 2 vectors of degree n in O(nlogn) , we can compute the vectors r and s in O(nlogn) time.
For any particular value of j in $F_j$ , we only have to compute one subtraction and two multiplication which is constant time. Hence computing $F_j$ for all values of j is O(n) time. \\
Therefore , total running time of the above algorithm is O(nlogn).


\end{document}
